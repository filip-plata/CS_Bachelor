\documentclass{beamer}

\usepackage{default}
\usepackage[MeX]{polski}
\usepackage[utf8]{inputenc}

\usetheme{Boadilla}
\usecolortheme{seahorse}

\usepackage{pgfpages}
%\setbeamertemplate{note page}[plain]
%\setbeameroption{show notes on second screen=right}
\setbeameroption{hide notes}
%\setbeameroption{show only notes}



\graphicspath{{../img/}}

% \ImgH{ile wysokości slajdu ma zajmować}{zdjęcie}
\newcommand{\ImgH}[2]{
	\begin{center}
		\includegraphics[height=#1\paperheight,keepaspectratio]{#2}
	\end{center}
}

% \ImgW{ile szerokości slajdu ma zajmować}{zdjęcie}
\newcommand{\ImgW}[2]{
	\begin{center}
		\includegraphics[width=#1\linewidth,keepaspectratio]{#2}
	\end{center}
}

\newcommand{\Title}[1]{\begin{center}{\LARGE \textbf{#1}}\end{center}}



\title{Hawala i Crowdfunding w Islamie}
%\subtitle{}
\author{Mateusz Banaszek, Filip Plata}
\date{6 listopada 2017}


\begin{document}

	\frame{ \titlepage
	}

%====================================================================
\section{Hawala}
	\frame{
		\Title{Hawala}
	}

	\frame{	\frametitle{Jak robiono przelewy tysiąc lat temu?}
		Zabieranie znacznej ilości pieniędzy w podróż jest niebezpieczne
		\begin{itemize}
			\item bandyci na popularnych szlakach handlowych
			\item mogą zatonąć razem ze statkiem
			\item można je zgubić...
		\end{itemize}
	}

	\frame{ \frametitle{Rozwiązanie kupców muzułmańskich}
		Hawala - czyli przeniesienie, zaufanie.
		Umożliwia dostęp do środków bez fizycznego transportu.
		Oparta na sieci społecznej ludzi zwanych "hawaladars" - w
		przeszłości często członków rodziny, klanu, grupy
		etnicznej.
		Zjawisko nie jest małe - w Somalii odpowiada za
		przepływ około 1.6 miliarda dolarów rocznie.
	}

	\frame{ \frametitle{Jak to działa}
		Klient A zgłasza się do lokalnego hawaladara i daje
		mu pieniądze, które chce przekazać osobie B w innym
		miejscu, najczęściej razem z hasłem.
		Hawaladar kontaktuje się innym hawaladarem
		(obecnie przez telefon, fax, email), który da pieniądze
		odbiorcy, gdy ten się zjawi i poda hasło.
	}

	\frame{ \frametitle{Obrazek działania}
		\ImgH{0.8}{hawala}
	}

	\frame{ \frametitle{Jak rozliczają się halawadarzy?}
		\begin{itemize}
			\item Transakcje w obie strony
			\item W przypadku  wyraźnego przepływu gotówki w jedną
			stronę (Indie - UK) - towarami
		\end{itemize}
	}

	\frame{ \frametitle{Przykładowa firmy}
			Dahabshiil - \url{https://www.dahabshiil.com}
			\ImgH{0.8}{edahab-picture}
	}

	\frame{ \frametitle{Zalety Hawali}
		\begin{itemize}
			\item Tania - opłaty około 0.2\% do 0.5\% pomiędzy miastami, 6\% - jedna z większych firm za przelew do dowolnego miejsca (24 000 do wyboru)
			\item Hawaladarzy mogą rozliczyć się jak chcą -
			mogą stosować realne zamiast oficjalnych kursów
			wymiany walut
			\item Można dotrzeć tam, gdzie przelew bankowy nie
			dotrze - Somalia, część Afganistanu, niektóre kraje
			Afryki
			\item Brak regulacji prawnych, opiera się tylko na
			dogadaniu się dwóch podmiotów - w przypadku mniejszych
			hawaladars, często ludzi
			\item Dodatkowe zajęcie dla lokalnych biznesów
			\item W wielu krajach nieopodatkowane
			\item Niski koszt - telefon, email
		\end{itemize}
	}

	\frame{ \frametitle{I wady}
		\begin{itemize}
			\item Oparcie tylko na zaufaniu
			\item Umożliwia pranie pieniędzy (piraci z Somalii podobno wysyłali tak dużo pieniędzy do Kenii)
			\item Niektóre zachodnie rządy (USA) patrzą bardzo podejrzliwie na taką działalność - zamykanie większych firm tego typu, zatrzymywanie ich pracowników jako terrorystów (Al-Barakat)
			\item Często brak żadnej kontroli, może być bardzo wygodne do prania pieniędzy
		\end{itemize}
	}

	\frame{ \frametitle{Podsumowanie}
		Praktyczny i działający system w rejonach świata, gdzie
		brak jest rozwiniętego systemu bankowego.
		Pralnia pieniędzy (?)
	}


%====================================================================
\section{Crowdfunding}
	\frame{
		\Title{Instrumenty finansowe / umowy / metody}
	}

	\frame{ \frametitle{Problem}
		Dlaczego nie ,,zachodnie'' rozwiązania?
		\begin{itemize}
			\item odsetki
			\item oprocentowanie
			\item lichwa
			\item ustalony stały zwrot inwestycji
			\bigskip
			\item Przedmiot/towar/usługa/\ldots zgodna z Szariatem
		\end{itemize}
	}

	\frame{ \frametitle{Murabaha}
		\begin{itemize}
			\item Murabaha = cost plus profit
			\item ,,Zakup na kredyt''
			\item Np. bank i klient
			\item Umowy:
				\begin{itemize}
					\item Zamówienie z przyrzeczeniem odkupu
					\item 2 kontrakty sprzedaży
				\end{itemize}
			\item Warunki, m.in.:
				\begin{itemize}
					\item Sprzedawca ma przedmiot w momencie zawarcia umowy
					\item Ustalona i niezmienna cenna
					\item Ustalony czas dostawy
					\item Ustalony harmonogram spłat
				\end{itemize}
		\end{itemize}
	}

	\frame{ \frametitle{Murabaha}
		\ImgH{0.8}{Murabaha_Brochure_002}
	}


	\frame{ \frametitle{Mudharabah}
		\begin{itemize}
			\item Mudharabah = profit-sharing arrangement
			\item Umowa między:
				\begin{itemize}
					\item inwestor: pieniądze
					\item przedsiębiorca: wiedza, umiejętności
				\end{itemize}
			\item Podział zysku ustalony na początku (np. 50:50, 60:40)
			\item A jak się nie uda?
				\begin{itemize}
					\item inwestor: traci pieniądze
					\item przedsiębiorca: czas, pracę, renomę, \ldots
				\end{itemize}
			\item \textbf{Przedsiębiorca nie gwarantuje zysku inwestorowi}\\ -- wszyscy dzielą ryzyko
		\end{itemize}

	}

	\frame{ \frametitle{Mudharabah}
		\ImgH{0.8}{Mudaraba}
	}

	\frame{ \frametitle{Musharakah}
		\begin{itemize}
			\item Musharakah = joint enterprise
			\item Podział zysku i strat
		\end{itemize}
		\ImgH{0.7}{Musharaka2}

	}

	\frame{
		\Title{Crowdfunding w Islamie}
	}

	\frame{ \frametitle{Rodzaje crowdfundingu}
		\begin{itemize}
			\item \underline{Lending-based (P2P)}
			\item Donation-based
			\item Reward-based
			\item Equity-based
		\end{itemize}
	}

	\frame{ \frametitle{Lending-based zgodne z Szariatem}
		\begin{itemize}
			\item Bez oprocentowania
			\item Inwestor daje pieniądze
			\item Przedsiębiorca daje wiedzę, pracę
			\item Inwestor (może) zarobi
			\item Przedsiębiorca nie oddaje praw, udziałów itp.
		\end{itemize}
	}

	\frame{ \frametitle{Lending-based zgodny z Szariatem}
		\begin{center}
			Potrzebujemy zebrać kapitał na nasze przedsięwzięcie?

			\textbf{Mudharabah (profit-sharing arrangement)}
		\end{center}
	}

	\frame{ \frametitle{www.ethiscrowd.com}
		\ImgH{0.8}{PROJECT-DEVELOPER-1}

	}
	\frame{ \frametitle{www.ethiscrowd.com}
		\ImgH{0.8}{fsac-dhariah-commitee2}

	}


	\frame{ \frametitle{Lending-based zgodny z Szariatem}
		\begin{center}
			Potrzebujemy zebrać pieniądze na zakupy dla naszego przedsięwzięcia?

			\textbf{Murabaha (cost plus profit)}
		\end{center}
	}


	\frame{ \frametitle{www.beehive.ae}
		\ImgH{0.8}{Tradeflow-chart}
	}

	\frame{ \frametitle{www.beehive.ae}
		\ImgW{0.8}{certificate}
	}

	\frame{ \frametitle{www.beehive.ae}
		\ImgW{1}{supervisory_board}
	}

	\frame{ \frametitle{kapitalboost.com}
		\ImgW{1}{new_murabaha}

	}

	\frame{ \frametitle{kapitalboost.com}
		\ImgW{1}{press_release}

	}

	\frame{ \frametitle{www.liwwa.com}
		\ImgW{1}{liwwa_faq}
	}


%====================================================================
\section{Koniec}
	\frame{	\frametitle{}
		\begin{center}
			\Huge{Pytania?}
		\end{center}
	}

	\frame{	\frametitle{Polecamy}
		Grafiki i informacje:
		\begin{itemize}
			\item \url{https://en.wikipedia.org/wiki/Profit_and_loss_sharing}
			\item \url{https://mfiles.pl/pl/index.php/Murabaha}
			\item \url{https://www.dmcc.ae/gateway-to-trade/financial-services/tradeflow}
			\item \url{www.ethiscrowd.com}
			\item \url{kapitalboost.com}
			\item \url{www.beehive.ae}
			\item \url{https://en.wikipedia.org/wiki/Hawala}

		\end{itemize}
	}

\end{document}
