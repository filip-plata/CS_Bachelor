\documentclass{beamer}

\usepackage{default}
\usepackage[MeX]{polski}
\usepackage[utf8]{inputenc}

\usetheme{Boadilla}
\usecolortheme{seahorse}

\usepackage{pgfpages}
%\setbeamertemplate{note page}[plain]
%\setbeameroption{show notes on second screen=right}
\setbeameroption{hide notes}
%\setbeameroption{show only notes}



\graphicspath{{../img/}}

% \ImgH{ile wysokości slajdu ma zajmować}{zdjęcie}
\newcommand{\ImgH}[2]{
	\begin{center}
		\includegraphics[height=#1\paperheight,keepaspectratio]{#2}
	\end{center}
}

% \ImgW{ile szerokości slajdu ma zajmować}{zdjęcie}
\newcommand{\ImgW}[2]{
	\begin{center}
		\includegraphics[width=#1\linewidth,keepaspectratio]{#2}
	\end{center}
}

\newcommand{\Title}[1]{\begin{center}{\LARGE \textbf{#1}}\end{center}}



\title{Kazaa, Skype i Torrent}
%\subtitle{}
\author{Mateusz Banaszek, Filip Plata}
\date{18 grudnia 2017}


\begin{document}

	\frame{ \titlepage
	}

%====================================================================
\section{Torrent i Skype}
	\frame{
		\Title{Torrent}
	}

	\frame{	\frametitle{Co to jest Torrent}
		Co to jest Torrent, wie chyba każdy -- plik z metadanymi
		dla protokołu BitTorrent.

		Należy odróżnić sam protokół od jego najczęstszych
		zastosowań.
	}

	\frame{ \frametitle{Protokół BitTorrent}
		\begin{itemize}
			\item P2P file sharing
			\item Jedno z głównych źródeł ruchu w internecie
			\item Wiele implementacji
			\item Miesięczna liczba użytkowników -- 1 miliard (2013)
			\item Specyfikacja - http://jonas.nitro.dk/bittorrent/bittorrent-rfc.html
		\end{itemize}
	}

	\frame{ \frametitle{Jak działa BitTorrent?}
		Torrent zawiera opis pliku, w szczególności podział na
		segmenty.

		Klient po pobraniu jakiegoś segmentu (1MB najczęściej) może potem sam go udostępniać innym. Duże pliki mogą być pobierane równolegle od różnych węzłów w sieci go udostępniających, co pozwala na znacznie szybsze pobieranie.

		Kolejność pobierania segmentów nie ma znaczenia.

		Informacje o seederach pobiera się z trackera wyspecyfikowanego w pliku .torrent
	}

	\frame{ \frametitle{Zalety}
		\begin{itemize}
			\item Znaczne zmniejszenie zapotrzebowania na zasoby
			podmiotu udostępniającego plik jako Torrent
			\item Szybsze, bo równoległe pobieranie segmentów
			\item Seederzy mogą być dużo bliżej niż oryginalny serwer
		\end{itemize}
		Blizzard rozprowadza w ten sposób np. patche
	}

	\frame{ \frametitle{Wady?}
		Brak mechanizmu do indeksowania torrentów
	}

	\frame{ \frametitle{Problemy prawne}
		Wiele Torrentów zawiera pliki znajdujące się tam bez
		zgody ich twórców.

		Można odtworzyć, kto pobierał lub udostępniał pliki.

		The Pirate Bay...
	}


	\frame{
		\Title{Skype}
	}

	\frame{ \frametitle{Trivia}
		Co to jest Skype również wszyscy raczej wiedzą.

		Rozmowy z jednego urządzenia na dowolne inne.

		Ponad 40\% rozmów międzynarodowych.

		Zamknięty protokół Microsoftu.

		Zamknięte oprogramowanie.

		Wspiera tylko IPv4.

		Można dzwonić w niektórych krajach na zwykłe telefony.
	}

	\frame{ \frametitle{Jak działa?}
		\begin{itemize}


		\item Dokładnie nie wiadomo. Zachowuje się podobnie do Kazzy, co nie dziwi, gdyż zostały stworzone przez tych samych ludzi.
		\item VOIP
		\item Sieć składa się z supernodów i zwykłych
		\item Automatyczna elekcja na supernode, najprawdopodobniej trzeba mieć publiczne IP
		\item Zwykłe nody rozpoczynają komunikację poprzez supernody (problem NATu)
		\end{itemize}
	}

	\frame{ \frametitle{Prywatność?}
		Microsoft udostępnia wszystkie dane rozmów (ich zawartość również) dla NSA i FBI.
	}

	\frame{ \frametitle{Skype w Chinach}
		Nie występuje.

		Chińczycy posiadają własny odpowiednik, stworzony przez TOM. Umożliwia to cenzorowanie treści, oraz ich analizowanie -- głównie pod kątem rozmów o Tajwanie, Tybecie, czy KPCh.
	}
%====================================================================
\section{Kazza}
	\frame{
		\Title{Kazaa}

		\begin{center}
			\ImgH{0.20}{Kazaa}
			{\Tiny en.wikipedia.org/w/index.php?curid=20669083}
		\end{center}
	}

	\frame{ \frametitle{Kazaa}
		\begin{itemize}
			\item P2P, protokół FastTrack
			\item Sharman Networks
			\item zamknięte oprogramowanie i protokół
			\item pobieranie klientów: 3.2M/tydzień (luty 2003)
			\item www.kazaa.com nie działa od sierpnia 2012
			\item zdjęcia, \underline{muzyka}, filmy, dokumenty
			\item adware/spyware
		\end{itemize}
	}

	\frame{ \frametitle{FastTrack}
		\begin{itemize}
			\item Kazaa, Grokster, iMesh, Morpheus
			\item 2.4M użytkowników na raz (2003)
			\item pobieranie kawałków pliku z różnych miejsc jednocześnie
			\item UUHash
			\item superwęzeł -- moc CPU, przepustowość łącza (dynamicznie)
			\item zamknięto-źródłowe szyfrowanie
			\item pobieranie danych po HTTP
			\item query flooding
			\item reverse-engineering protokołu klient---superwęzeł
		\end{itemize}
	}

	\frame{ \frametitle{Analiza ruchu}
		\textit{Deconstructing the Kazaa Network}\\
		Nathaniel Leibowitz, Matei Ripeanu, Adam Wierzbicki

		2003

		\vspace{2em}

		Analiza ruchu Kazaa (pobieranie po HTTP) u izraelskiego ISP.
	}

	\frame{ \frametitle{Zebrane statystyki}
		\ImgW{1}{table1}
	}

	\frame{ \frametitle{Popularne pliki -- pobrania}
		\ImgH{0.8}{fig1}
	}

	\frame{ \frametitle{Popularne pliki -- pobrania}
		\ImgH{0.8}{fig2}
	}

	\frame{ \frametitle{Popularne pliki -- ruch}
		\ImgH{0.8}{fig3}
	}

	\frame{ \frametitle{Popularne pliki -- ruch}
		\ImgH{0.8}{fig4}
	}

	\frame{ \frametitle{Wielkość plików}
		\ImgH{0.7}{fig5}

		\begin{itemize}
			\item $>=$ 700MB --- 60\% ruchu
		\end{itemize}
	}

	\frame{ \frametitle{Nowe pliki}
		\ImgH{0.8}{fig8}
		% Once a file is requested it will be requested again soon
	}

	\frame{ \frametitle{Nowe pliki}
		\ImgH{0.8}{fig10}
	}

	\frame{ \frametitle{Popularność plików -- stałość}
		\ImgH{0.8}{fig11}
		% 2: most likely the Kazaa software installation packages
	}

	\frame{ \frametitle{Popularność plików -- stałość}
		\ImgH{0.65}{fig12}

		\begin{itemize}
			\item zawsze popularne
			\item krótko popularne
		\end{itemize}
	}

	\frame{ \frametitle{data-sharing graph}
		\begin{itemize}
			\item węzeł: użytkownik
			\item krawędź: podobieństwo pobrań\\
				$>= m$ wspólnych plików w okresie $T$
			\item $m = $ 1..5
			\item $T = $ 4..48 godzin
		\end{itemize}

		\vspace{2em}

		small-world network
		\begin{itemize}
			\item mała średnia długość ścieżki (jak w losowym grafie)
			\item znacznie większy clustering coefficient niż w losowym grafie
		\end{itemize}
	}

	\frame{ \frametitle{data-sharing graph}
		\ImgH{0.8}{table2}
	}

	\frame{ \frametitle{data-sharing graph}
		\ImgH{0.8}{fig14}
	}


%====================================================================
\section{Koniec}
	\frame{	\frametitle{}
		\begin{center}
			\Huge{Pytania?}
		\end{center}
	}

	\frame{	\frametitle{Polecamy}
		Grafiki i informacje:
		\begin{itemize}
		\item \url{https://en.wikipedia.org/wiki/BitTorrent}
		\item \url{http://www.bittorrent.com/lang/pl/}
		\item \url{https://en.wikipedia.org/wiki/Skype}
		\item \url{http://www.divms.uiowa.edu/~ghosh/skype.pdf}
		\item \url{https://en.wikipedia.org/wiki/Kazaa}
		\item \url{https://en.wikipedia.org/wiki/FastTrack}
		\item \url{https://pl.wikipedia.org/wiki/FastTrack}
		\item \url{https://www.globus.org/sites/default/files/kazaa.pdf}
		\end{itemize}
	}

\end{document}
