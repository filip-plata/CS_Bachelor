\documentclass[12pt]{article}

\usepackage{verbatim}
\usepackage{syntax}
\usepackage{listings}
\usepackage[T1]{fontenc}
\usepackage[polish]{babel}
\usepackage[utf8]{inputenc}
\usepackage{lmodern}
\selectlanguage{polish}

\title{Projekt na przedmiot Języki i Narzędzia Programowania}
\author{
        Filip Plata \\
        Uniwersytet Warszawski
}
\date{\today}

\begin{document}
\maketitle

\section{Twórcy}

\begin{itemize}
\item Filip Plata
\end{itemize}

\section{Opis projektu}

Aplikacja pozwalająca na reagowanie na wydarzenia na giełdzie - o roboczej nazwie stockalert.

Głównie pomyślana w tym momencie właśnie do notyfikacji w reakcji na zmiany cen akcji, wedle zdefiniowanych alertów.

Wstępnie do notyfikowania poprzez wysyłanie emaili w reakcji na przekroczenie przez papier wartościowy pewnej górnej lub dolnej granicy.
Możliwy kierunek rozszerzenia to notyfikacje na urządzenia mobilne.

Te funkcjonalności powyżej, choć proste, już nie są proste do uzyskania z darmowych narzędzi. Również typy zleceń na giełdę z aktywacją cenową są znacznie mniej elastyczne niż sama notifikacja, bo nie uwzględniają dynamiki cen.

Oprócz tego, architektura jest pomyślana rozszerzalnie, może dojść więcej serwisów procesujących ceny niż samo sprawdzanie górnych/dolnych granic.

\section{Architektura}

Aplikacja podzielona jest na trzy mikroserwisy, korzystające ze wspólnej (w sensie instancji serwera) bazy danych. Do części komunikacji wewnętrznej wykorzystywany jest RabbitMQ.

Notifier - zajmuje się notyfikowaniem użytkownika o wykrytych zdarzeniach. W pierwszej iteracji będzie wysyłał email. Komunikuje się poprzez RabbitMQ.

Thresholdalert - jest informowany o nowych cenach na giełdzie i przetwarza je pod kątem spadku/wzrostu przebijającego zdefiniowane przez użytkonika poziomy. Informowanie o nowych danych odybwa się poprzez RabbitMQ, następnie wyceny są pobierane po HTTP.

StockData - pobiera dane z jednego ustalonego źródła i zapisuje je do bazy. Po pobraniu danych informuje mikroserwis(y) przetwarzające o nowych danych i podaje ich lokację (czyli URL). Oprócz tego posiada dane o użytkownikach.

Wszystkie dane mikroserwisów są od siebie logicznie oddzielone.

Nie bedzie żadnych zabezpieczeń API - dałoby to tylko narzut implementacyjny.

\section{Technologie}

Do implementacji serwisów zostanie użyty Python, ze względu na łatwość i szybkość implementacji. Była rozważana Akka (Http), jednakże wtedy program wyszedłby mało `mikroserwisowy`.

ThresholdAlert ma być we Flasku, a StockData w Django wraz z licznymi pod-frameworkami. Wybór dobrany był pod stopień złożoności zadań wykonywanych przez serwisy.

Bazą danych będzie Postgres, gdyż z moich doświadczeń to najbardziej solidny darmowy (a pewnie i nie tylko) RDBMS, a także gdyż jest dostępny na AWS. Każdy serwis bedzie w dockerze. Całość zostanie docelowo uruchomiona na AWS.

RabbitMQ jest dla dodania asynchroniczności i lepszego rozdzielenia działalności serwisów od siebie.

Frontend nie jest przewidziany, aczkolwiek może zostać dopisany - wtedy najprawdopodobniej byłby oparty o React - a na pewno tylko o komunikację REST z mikroserwisami.


\end{document}
