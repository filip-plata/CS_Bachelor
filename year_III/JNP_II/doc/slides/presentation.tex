\documentclass[serif,mathserif]{beamer}
\usepackage{amsmath, amsfonts, epsfig, xspace}
\usepackage{algorithm,algorithmic}
\usepackage{pstricks,pst-node}
\usepackage{multimedia}
\usepackage{polski}
\usepackage[utf8]{inputenc}
\usepackage[normal,tight,center]{subfigure}
\setlength{\subfigcapskip}{-.5em}
\usepackage{beamerthemesplit}
\usetheme{lankton-keynote}

\author[Filip Plata]{Filip Plata}

\title[Short Title\hspace{2em}\insertframenumber/\inserttotalframenumber]{Stocktick}

\institute{Uniwersytet Warszawski}

\begin{document}

\maketitle

% \section{Introduction}

\begin{frame}
  \frametitle{Wstęp}
  \pause
  \begin{itemize}
  \item Czy to robi coś przydatnego?\pause
  \item Czy warto używać mikroserwisów dla takich projektów?\pause
  \item Architektura i komunikacja
  \end{itemize}
\end{frame}

% \section{Main Body}

\begin{frame}
  \frametitle{Co robi stocktick?}
  Prosta infrastruktura do wykonywania kawałków kodu
  w reakcji na zmiany cen papierów wartosciowych.
  Planowane zszycie się z kodem kolegi, jednak na ten moment:
  \begin{itemize}
  \item Tylko akcje\pause
  \item Tylko okresowe wydarzenia \pause
  \item Tylko reakcja na zbyt niską/wysoką cenę\pause
  \item Tylko wysyła emaile\pause
  \item Żadnege frontendu i zabezpieczeń
  \end{itemize}
\end{frame}

\begin{frame}
  \frametitle{Architektura}
  Cztery mikroserwisy:
  \begin{itemize}
  \item Credentials
  \item Notify
  \item Thresholdalert
  \item stocktick-django
  \end{itemize}
\end{frame}

\begin{frame}
  \frametitle{Komunikacja}
  Cztery mikroserwisy:
  \begin{itemize}
  \item Credentials - HTTP, informacje o użytkownikach
  \item Notify - AMPQ, synchroniczność jest niepotrzebna
  \item Thresholdalert - HTTP tworzenie alertów, AMPQ przetwarzanie zdarzeń na giełdzie
  \item stocktick-django - HTTP do zarządzania danymi, generowanie okresowych zdarzeń do przetworzenia przez resztę systemu - AMPQ
  \end{itemize}
\end{frame}

\begin{frame}
  \frametitle{Demo}
  Ręczne odpalanie przetworzenia danych z giełdy na systemie uruchomionym na AWS.
\end{frame}

% \section{Conclusion}

\begin{frame}
  \frametitle{Pytania}
\end{frame}
\end{document}
