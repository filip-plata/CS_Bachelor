\documentclass[11pt]{article}

\usepackage{polski}
\usepackage[utf8]{inputenc}
\usepackage{floatflt,amsmath,amssymb}
\usepackage[ligature, inference]{semantic}

\title{Semantyka praca domowa}
\author{Filip Plata}
\date{\today}


\begin{document}


\maketitle

\mathlig{->}{\rightarrow}
\mathlig{|-}{\vdash}
\mathlig{=>}{\Rightarrow}
\mathligson

\section{Opis}

Podczas definiowania semantyki będziemy korzystać ze zbiorów

\begin{itemize}
\item LOC czyli zbioru lokacji na zmienne
\item Env = VAR $\rightarrow$ LOC, środowisko
\item CatchB = $\mathcal{P}(BExpr \times Env)$, skonczony zbiór wyrażeń catch wraz ze środowiskiem w którym zostały zdefiniowane
\item S = Loc $\rightarrow \mathbb{N}$, czyli stan
\item $I \times Env \times CatchB \times S \rightarrow S$, czyli relacja przejścia
\end{itemize}

Oprócz tego będę używał funkcji \textit{newloc} ze standardowym znaczeniem oraz funkcji 
$\textit{checkAll} : CatchB \rightarrow S \rightarrow Bool$ która sprawdza, czy któryś z warunków logicznych jest spełniony dla środowiska, w którym został on zdefiniowany.

Podczas pisania semantyki będziemy utrzymywać przydatny warunek, iż jeżeli wejdziemy do jakiejś instrukcji, to mamy ją wykonać (to oczywiście z punktu widzenia małych kroków, to stwierdzenie jest bardziej intuicyjne niz ścisłe). O to, by nie wykonać za dużo, będziemy dbać w regułach dla while i średnika.

\section{Reguły semantyki}

\begin{itemize}
\item $<skip, \rho, Q, s> \longrightarrow s$

\item $<x := e, \rho, Q, s> \longrightarrow s'$, gdzie $e$ ewalujuje sie do $n$, $\rho(x) = l$ oraz $s' = s[l\rightarrow n]$

\item $<if \  b \  then \   I_{1} \    else \  I_{2}, \rho, Q, s>$
, ewauluje się do wynku ewualuacji pierwszej z gałęzi lub drugiej, zależnie od wartości semantycznej \textit{b}, ze stanem, instrukcją, środowiskiem i zbiorem \textit{Q} przepisanymi wprost.
Czyli w najbardziej ze standardowych sposobów.

\item Dla instrukcji \textit{catch} jeśli:

\begin{itemize}
\item $<I, \rho, Q \cup \{(b, \rho)\}, s> \longrightarrow s'$
\end{itemize}

Wtedy $<catch \ b \  in \ I, \rho, Q, s> \longrightarrow s'$

\item Semantyka deklaracji jest standardowa, jeśli:

\begin{itemize}
\item $<D, \rho, s> \longrightarrow (\rho', s')$
\item $<I, \rho, Q, s'> \longrightarrow s''$
\end{itemize}

to $<begin \ D;I \ end, \rho, Q, s> \longleftrightarrow s''$

\item Dla średnika sprawdzamy, czy po wykonaniu pierwszej instrukcji nie musimy przerwać obliczeń.

\begin{itemize}
\item $<I_{1}, \rho, Q, s> \rightarrow s'$
\item $checkAll(Q, s') = true$
\end{itemize}

To $<I_{1};I_{2}, \rho, Q, s> \longrightarrow s'$. Z kolei gdy:

\begin{itemize}
\item $<I_{1}, \rho, Q, s> \rightarrow s'$
\item $<I_{2}, \rho, Q, s'> \rightarrow s''$
\item $checkAll(Q, s') = false$
\end{itemize}

To $<I_{1};I_{2}, \rho, Q, s> \longrightarrow s''$

\item Jeśli powyżej \textit{checkAll} zwróci fałsz, to bierzemy stan \textit{s'} z ewaluacji pierwszej instrukcji i mając $<I_{2}, \rho, Q, s> \rightarrow s''$
ewalujemy średnik do \textit{s''}

\item Dla instrukcji \textit{while} jeśli warunek logiczny nie jest spełniony: przepisujemy niezmineiony stan

\item Instrukcja \textit{while} gdy warunek jest spełniony
\begin{itemize}
\item $<I, \rho, Q, s> \longrightarrow s'$
\item $checkAll(Q, s) \longrightarrow true$
\end{itemize}

Wtedy, ponieważ zaszedł warunek z catch:

$<while \ b \  do \ I, \rho, Q, s> \longrightarrow s'$

Natomiast jeśli sprwadzenie dało false, mamy regułę

\begin{itemize}
\item $<I, \rho, Q, s> \longrightarrow s'$
\item $checkAll(Q, s) \longrightarrow false$
\item $<while \ b \  do \ I, \rho, Q, s'> \longrightarrow s''$
\end{itemize}

To: $<while \ b \  do \ I, \rho, Q, s> \longrightarrow s''$

\end{itemize}

Taka semantyka nie jest optymalna pod wzgledem implementacji. Możnaby wykonywać mniej spradzeń warunków z instrukcji \textit{catch}, gdyby sprwadzać wszystkie warunki z catch tylko po przypisaniach.


\end{document}